\documentclass[12pt]{article}

\usepackage[utf8]{inputenc}

%%% PAGE DIMENSIONS
\usepackage{geometry}
\geometry{letterpaper}

\usepackage{graphicx}

%%% PACKAGES
\usepackage{listings}
\usepackage{color}
\usepackage{booktabs}
\usepackage{array}
\usepackage{verbatim}
\usepackage{subfig}

%%% HEADERS & FOOTERS
\usepackage{fancyhdr}
\pagestyle{fancy}
\renewcommand{\headrulewidth}{0pt}
\lhead{}\chead{}\rhead{}
\lfoot{}\cfoot{\thepage}\rfoot{}

%%% END Article customizations

\title{Connect 4 Terminal Game Documentation}
\author{Giancarlo Pernudi}

\definecolor{codegreen}{rgb}{0,0.6,0}
\definecolor{codegray}{rgb}{0.5,0.5,0.5}
\definecolor{codepurple}{rgb}{0.58,0,0.82}
\definecolor{backcolour}{rgb}{0.95,0.95,0.92}

\lstdefinestyle{mystyle}{
    backgroundcolor=\color{backcolour},
    commentstyle=\color{codegreen},
    keywordstyle=\color{magenta},
    numberstyle=\tiny\color{codegray},
    stringstyle=\color{codepurple},
    basicstyle=\footnotesize,
    breaklines=true,
    keepspaces=true,
    numbersep=5pt,
}
\lstset{style=mystyle}

\begin{document}
\maketitle

\section{Board Class}

This class is the board, and where all the interaction will flow through. It has the following variables:
\begin{itemize}
  \item cells: a 2d array of integers representing the current game state (6x7)
  \item turn: dictates which player's turn it is
  \item characters: a dictionary for characters correspoding to an integer (0-2)
\end{itemize}


The Class has the following methods:
\begin{lstlisting}[language=Python]
displayPlayer() # returns formatted of information of the current player's turn and their token character
displayBoard() # prints out a graphical representation of the board
getBoard() # returns the cells
changeTurn() # alternates the turn between 1 and 2
getTurn() # returns the current turn
play() # takes as input the index of column and plays
isFull() # returns true if board is full
restart() # resets the board for a new game
checkWin() # checks for possible win by calling the following three methods:
rowWin() # horizontal
colWin() # vertical
diagWin() # both diagonal directions
colNotFull() # return a list of column indexes which are not full
\end{lstlisting}

\pagebreak

\section{HumanPlayer Class}

%This class is an actual human player. It has the following variables:
%\begin{itemize}
%  \item can't think of a necessary variable atm
%\end{itemize}

This class is an actual human player. The Class has the following methods:
\begin{lstlisting}[language=Python]
getInput() # asks for a play from the player and will only accept a valid response. Will ask again if necessary. 
\end{lstlisting}

\section{DumbBot Class}

This class is the computer player, with a random algorithm. It has the following variables:
\begin{itemize}
  \item difficulty: how smart or dumb this computer will play
\end{itemize}

The Class has the following methods:
\begin{lstlisting}[language=Python]
getPlay() # tries to pick the best possible move
\end{lstlisting}


\section{SmartBot Class}

This class is the computer player, with a minmax algorithm. It has no variables.

The Class has the following method:
\begin{lstlisting}[language=Python]
getPlay() # picks a random move from a list of valid moves
\end{lstlisting}

\end{document}
